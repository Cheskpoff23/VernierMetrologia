\documentclass{article}
%%%%%%%%%%%%%%%%%%%%%%%%%%%%% Define Article %%%%%%%%%%%%%%%%%%%%%%%%%%%%%%%%%%
%%%%%%%%%%%%%%%%%%%%%%%%%%%%%%%%%%%%%%%%%%%%%%%%%%%%%%%%%%%%%%%%%%%%%%%%%%%%%%%

%%%%%%%%%%%%%%%%%%%%%%%%%%%%% Using Packages %%%%%%%%%%%%%%%%%%%%%%%%%%%%%%%%%%
\usepackage{float}
\usepackage[letterpaper,portrait]{geometry}
\usepackage{graphicx}
\usepackage{anysize}
\usepackage{lipsum}
\usepackage{amsmath,amssymb,amsthm}
\usepackage[utf8]{inputenc}
\usepackage{multirow}
\usepackage{csquotes}
\usepackage[spanish]{babel}
\usepackage{apacite}
\usepackage{multicol}
\usepackage{parskip}
\usepackage{setspace}
\usepackage{empheq}
\usepackage{mdframed}
\usepackage{booktabs}
\usepackage{lipsum}
\usepackage{graphicx}
\usepackage{color}
\usepackage{psfrag}
\usepackage{pgfplots}
\usepackage{bm}
\usepackage{tocloft}
\usepackage{lscape}
\usepackage{adjustbox}
%%%%%%%%%%%%%%%%%%%%%%%%%%%%%%%%%%%%%%%%%%%%%%%%%%%%%%%%%%%%%%%%%%%%%%%%%%%%%%%

% Other Settings

%%%%%%%%%%%%%%%%%%%%%%%%%% Page Setting %%%%%%%%%%%%%%%%%%%%%%%%%%%%%%%%%%%%%%%
\geometry{letterpaper, margin=2.54cm}

%%%%%%%%%%%%%%%%%%%%%%%%%% Define some useful colors %%%%%%%%%%%%%%%%%%%%%%%%%%
\definecolor{ocre}{RGB}{243,102,25}
\definecolor{mygray}{RGB}{243,243,244}
\definecolor{deepGreen}{RGB}{26,111,0}
\definecolor{shallowGreen}{RGB}{235,255,255}
\definecolor{deepBlue}{RGB}{61,124,222}
\definecolor{shallowBlue}{RGB}{235,249,255}
%%%%%%%%%%%%%%%%%%%%%%%%%%%%%%%%%%%%%%%%%%%%%%%%%%%%%%%%%%%%%%%%%%%%%%%%%%%%%%%

%%%%%%%%%%%%%%%%%%%%%%%%%% Define an orangebox command %%%%%%%%%%%%%%%%%%%%%%%%
\newcommand\orangebox[1]{\fcolorbox{ocre}{mygray}{\hspace{1em}#1\hspace{1em}}}
%%%%%%%%%%%%%%%%%%%%%%%%%%%%%%%%%%%%%%%%%%%%%%%%%%%%%%%%%%%%%%%%%%%%%%%%%%%%%%%

%%%%%%%%%%%%%%%%%%%%%%%%%%%% English Environments %%%%%%%%%%%%%%%%%%%%%%%%%%%%%
\newtheoremstyle{mytheoremstyle}{3pt}{3pt}{\normalfont}{0cm}{\rmfamily\bfseries}{}{1em}{{\color{black}\thmname{#1}~\thmnumber{#2}}\thmnote{\,--\,#3}}
\newtheoremstyle{myproblemstyle}{3pt}{3pt}{\normalfont}{0cm}{\rmfamily\bfseries}{}{1em}{{\color{black}\thmname{#1}~\thmnumber{#2}}\thmnote{\,--\,#3}}
\theoremstyle{mytheoremstyle}
\newmdtheoremenv[linewidth=1pt,backgroundcolor=shallowGreen,linecolor=deepGreen,leftmargin=0pt,innerleftmargin=20pt,innerrightmargin=20pt,]{theorem}{Theorem}[section]
\theoremstyle{mytheoremstyle}
\newmdtheoremenv[linewidth=1pt,backgroundcolor=shallowBlue,linecolor=deepBlue,leftmargin=0pt,innerleftmargin=20pt,innerrightmargin=20pt,]{definition}{Definition}[section]
\theoremstyle{myproblemstyle}
\newmdtheoremenv[linecolor=black,leftmargin=0pt,innerleftmargin=10pt,innerrightmargin=10pt,]{problem}{Problem}[section]
%%%%%%%%%%%%%%%%%%%%%%%%%%%%%%%%%%%%%%%%%%%%%%%%%%%%%%%%%%%%%%%%%%%%%%%%%%%%%%%

%%%%%%%%%%%%%%%%%%%%%%%%%%%%%%% Plotting Settings %%%%%%%%%%%%%%%%%%%%%%%%%%%%%
\usepgfplotslibrary{colorbrewer}
\pgfplotsset{width=8cm,compat=1.9}
%%%%%%%%%%%%%%%%%%%%%%%%%%%%%%%%%%%%%%%%%%%%%%%%%%%%%%%%%%%%%%%%%%%%%%%%%%%%%%%

%%%%%%%%%%%%%%%%%%%%%%%%%%%%%%% Title & Author %%%%%%%%%%%%%%%%%%%%%%%%%%%%%%%%
\author{Gustavo Vergara}
%%%%%%%%%%%%%%%%%%%%%%%%%%%%%%%%%%%%%%%%%%%%%%%%%%%%%%%%%%%%%%%%%%%%%%%%%%%%%%%

\begin{document}
\pgfplotsset{compat=1.18}
\setstretch{2}

\begin{titlepage}
	\centering
	\vspace{2.5cm}
	{\scshape \Large Elaboración de los diagramas del modelo de dominio del proyecto - GA2-220501093-AA2-EV01 \par}
	\vspace{5cm}
	\textbf\large\scshape{\par}
	\vspace{0.5cm}

	{\Large Vergara Pareja Gustavo\par}
	\vspace{5cm}
	{\scshape\Large Jovanna Herazo\par}
	\vspace{0.3cm}
	{\scshape\Large Tecnología en Análisis y Desarrollo de Software \par}
	\vspace{0.3cm}
	{\scshape\Large SENA - Centro Agropecuario Regional Cauca\par}
	\vspace{0.3cm}
	{\Large \today \par}
\end{titlepage}
\tableofcontents
\newpage

	\large \textbf{ACTIVIDAD A SOLUCIONAR}\\
	\vspace{0.1cm}
	\section*{TALLER INSTRUMENTOS - METROLOGÍA Y CONTROL DE CALIDAD}
\section{¿Qué es el calibrador Vernier o pie de rey?}
El vernier es una escala auxiliar que se desliza a lo largo de una escala principal para permitir en ésta lecturas fraccionales exactas de la mínima división. 
Para lograr lo anterior, una escala vernier está graduada en un número de divisiones iguales en la misma longitud que n-1 divisiones de la escala principal; ambas escalas están marcadas en la misma dirección. Una fracción de 1/n de la mínima división de la escala principal puede leerse. Fué elaborado para satisfacer la necesidad 
 de un instru­mento de lectura directa que pudiera brindar una medida fácilmente, en una sola 
 operación. El calibrador típico puede tomar tres tipos de mediciones: exteriores, 
 interiores y profundidades, pero algunos además pueden realizar medición de peldaño. 
	\begin{itemize}
		\item ¿Cuántos modelos de vernier existen?
		\item ¿Para qué sirve cada uno?
		\item ¿Cuáles son sus partes?
		\item ¿Cómo se lee la medida en el instrumento?
	\end{itemize}
	\section{¿Qué es el micrómetro?}
	\begin{itemize}
		\item ¿Cuáles son sus partes?
	
		\item¿Cómo se lee la medida en el instrumento?
		
		\item¿Qué es el micrómetro?
	
		\item¿Cuántos modelos de micrómetros existen?
		
		\item¿Para qué sirve cada uno?
	
		\item¿Cuáles son sus partes?
	\end{itemize}

\newpage

\begin{figure}[H]
	\centering
	\includegraphics[width=1\textwidth]{partes_vernier.png}
	\caption{Diagrama de paquetes}
	\label{fig:imagen2}
\end{figure}
Es un instrumento de medición que se utiliza para medir longitudes, diámetros, profundidades
 y espesores con una precisión de hasta 0,02 mm. Fué elaborado para satisfacer la necesidad 
 de un instru­mento de lectura directa que pudiera brindar una medida fácilmente, en una sola 
 operación. El calibrador típico puede tomar tres tipos de mediciones: exteriores, 
 interiores y profundidades, pero algunos además pueden realizar medición de peldaño. 

\end{document}